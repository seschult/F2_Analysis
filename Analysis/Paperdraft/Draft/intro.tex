
%\section{Introduction}\label{sec:intro}
%\label{intro}

The top quark mass is one of the most important parameters in the
Standard Model (SM). As the top quark features the largest Yukawa coupling, 
it is closely linked to Higgs physics. Furthermore, 
the Higgs potential and therefore the vacuum stability
of the SM depends critically on the value of the top quark mass.
Processes involving top quarks allow for important precision tests of
the SM and appear amongst the dominant backgrounds for 
many New Physics searches. They also allow to further constrain the
gluon PDF at large $x$-values~\cite{Czakon:2013tha,Guzzi:2014wia,delDuca:2015gca,Czakon:2016olj}.

The measurement of the top quark mass is  complicated  due to the fact that
the reconstruction of $t\bar{t}$ events from complex hadronic and
leptonic final states is an arduous task.
Measurements of the top quark mass have been performed in various
channels by the Tevatron and LHC collaborations, where the latest
combinations can be found in Refs.~\cite{CDF-1402,Abazov:2017ktz,ATLAS-CONF-2017-071,Khachatryan:2015hba}.
While the most precise result in the di-lepton channel has an uncertainty of
$0.84\gev$~\cite{Aaboud:2016igd}, the most precise combined results for the
top quark mass achieve a precision of
about $0.5\gev$~\cite{Khachatryan:2015hba,ATLAS-CONF-2017-071}.
The precision achieved nowadays is the result of joint efforts in the experimental as well as the
theory community to reduce the systematic uncertainties inherent to top quark mass measurements.
%
For recent theoretical studies with regards to the definition and
extraction of the top quark mass, see e.g.~\cite{Frixione:2014ala,Beneke:2016cbu,Butenschoen:2016lpz,Kawabata:2016aya,Hoang:2017suc,Hoang:2017btd,Hoang:2017kmk,Bevilacqua:2017ipv,Corcella:2017rpt,Ravasio:2018lzi}.

The theoretical description of top quark pair production at hadron
colliders has improved substantially in recent years.
For stable top quarks, NNLO corrections to differential distributions are
known~\cite{Czakon:2015owf,Czakon:2016dgf,Czakon:2017dip} and have recently 
been combined with
NLO electroweak corrections~\cite{Czakon:2017wor}.
The impact of electroweak corrections on distributions related to
$t\bar{t}$ production has been studied in
Refs.~\cite{Hollik:2011ps,Kuhn:2013zoa,Pagani:2016caq} for on-shell top quarks, and
in Ref.~\cite{Denner:2016jyo} for both the on-shell case and with complete
off-shell effects. Electroweak corrections to multi-jet merged
on-shell top quark pair production have been calculated in Ref.~\cite{Gutschow:2018tuk}.
Due to their very high complexity, the NNLO fixed-order calculations have so far only
been combined with top quark decays in the narrow-width approximation
(NWA), which factorises the production and decay processes.
Radiative corrections to top quark decays have been calculated in
Refs.~\cite{Bernreuther:2004jv,Melnikov:2009dn,Campbell:2012uf}, and
since have been extended up to NNLO QCD~\cite{Brucherseifer:2013iv,Gao:2017goi}.
Resummation also has been accomplished up to NNLL, together with other
improvements going beyond fixed
order~\cite{Beneke:2011mq,Cacciari:2011hy,Ferroglia:2013awa,Broggio:2014yca,Kidonakis:2015dla,Pecjak:2016nee}. 


However, a description of top quark pair production and decay which
predicts the {\em shapes} of distributions with an accuracy required for
improvements on the current experimental precision needs to go beyond the
narrow-width approximation.
NLO QCD calculations of $W^+W^- b\bar{b}$ production, including leptonic
decays of the $W$ bosons,  have been performed in
Refs.~\cite{Denner:2010jp,Denner:2012yc,Bevilacqua:2010qb,Heinrich:2013qaa}. 
These calculations use the 5-flavour scheme, where
the $b$-quarks are treated as massless partons.
In Ref.~\cite{Heinrich:2013qaa}, particular emphasis has been put on the
impact of the non-factorising contributions on the top quark mass
measurements. 
Recently the calculation of the NLO QCD corrections to $W^+W^- b\bar{b}$ production with
full off-shell effects has also been achieved in the lepton plus jets channel~\cite{Denner:2017kzu}.

The $b$-quark mass effects on observables like the invariant mass of a
lepton-$b$-quark pair  ($m_{lb}$)  are very small. 
However, the use of 
massive $b$-quarks (more precisely, the 4-flavour scheme, 4FNS)  has the
(technically) important feature that it avoids collinear singularities
due to $g\to b\bar{b}$ splittings. This implies that any phase space restrictions on
the $b$-quarks can be made without destroying infrared safety, and
thus allows  to consider 0, 1- and 2-jet bins for 
$pp\to e^+\nu_e\mu^-\bar{\nu}_\mu b\bar{b}$ in one and the same setup, 
which is important for cross sections defined by jet vetos.
In Refs.~\cite{Frederix:2013gra,Cascioli:2013wga}, NLO calculations in the 4FNS have been performed.

The next step in complexity towards a realistic description of the
measured final states consists in combining fixed-order calculations with a parton shower. 
The effect of radiative corrections to both, production and decay, in the factorised
approach matched to a parton shower has been investigated in Ref.~\cite{Campbell:2014kua}
within an extension of the {\tt PowHeg}~\cite{Frixione:2007vw,Alioli:2010xd} framework, called {\tt ttb\_NLO\_dec}
in the {\tt POWHEG-BOX-V2}.
Within the {\tt Sherpa} framework, NLO QCD predictions for top quark
pair production with up to three jets matched to a parton shower are
also available, see Refs.~\cite{Hoeche:2014qda,Hoche:2016elu}.
A new NLO multi-jet merging algorithm relevant to top quark pair
production is also available in {\tt Herwig\,7.1}~\cite{Bellm:2017idv}.

Based on an NLO calculation of $W^+W^- b\bar{b}$ production combined
with the {\tt Powheg} framework, first results of the $W^+W^-
b\bar{b}$ calculation in the 5-flavour scheme matched to a parton
shower have been presented in Ref.~\cite{Garzelli:2014dka}. 
However, it has been noticed later that the matching of NLO matrix elements
involving resonances of coloured particles to parton showers poses
problems which can lead to artefacts in the top quark lineshape~\cite{Jezo:2015aia}.
As a consequence, an improvement of the resonance treatment has been
implemented in {\tt POWHEG-BOX-RES}, called ``resonance aware matching'', 
and combined with NLO matrix elements from OpenLoops~\cite{Cascioli:2011va}, to arrive at 
the most complete description so far~\cite{Jezo:2016ujg}, based on the
framework developed in Ref.~\cite{Jezo:2015aia} and the 4FNS calculation of
Ref.~\cite{Cascioli:2013wga}. An alternative algorithm to treat
radiation from heavy quarks in the {\tt Powheg} NLO+PS framework has been
presented in Ref.~\cite{Buonocore:2017lry}.
An improved resonance treatment in the matching to parton showers for off-shell single top production at NLO
has been worked out in Ref.~\cite{Frederix:2016rdc}, and similarly for off-shell
$t\bar{t}$ and $t\bar{t}H$ production in $e^+e^-$ collisions in Ref.~\cite{Nejad:2016bci}.

\medskip

In this paper, we investigate the impact of different
approximations on the top quark mass measurement simulating a concrete experimental setup. 
In particular, we follow up on an open question raised in Ref.~\cite{Heinrich:2013qaa}, where 
we performed a study of NLO effects in top quark mass
measurements based on the observable \mlb\ in the framework of a top quark mass measurement as performed by ATLAS
using the template method~\cite{Aad:2015nba,Aaboud:2016igd}. 
Substantial distortions in the \mlb\ distribution are induced by scale variations 
calculated by including  the full NLO corrections to the $W^+W^- b\bar{b}$ final state
(with leptonic $W$-decays). On the other hand, in the factorised
approach, where the $t\bar{t}$ cross section calculated at NLO is
combined with  LO top quark decays in the NWA, the shape distortions
due to the scale variations  are minor. 
As the experimental analysis is based on normalised distributions, the
shape differences induced by scale variations translate in a very
sensitive manner into the theoretical uncertainties on the extraction
of the top quark mass.

The question arises where the shape changes come from, i.e. whether they mainly
come from the non-factorisable contributions contained in the full NLO
corrections to $W^+W^- b\bar{b}$, or from factorisable NLO corrections to the top quark decay. 
And, if the latter is true, what is the effect of a parton shower in
combination with the factorised approach, as it should contain the
leading contributions of the NLO corrections to the top
quark decay.
To answer these questions, we compare the NLO calculation of $W^+W^- b\bar{b}$
production of Ref.~\cite{Heinrich:2013qaa} with the calculation based
on the narrow-width approximation 
where both $t\bar{t}$ production {\it and} decay are calculated at NLO,
as described in Ref.~\cite{Melnikov:2009dn}.
We further quantify the impact of a parton shower in the narrow-width
approximation, combining the NLO matrix elements of top quark pair
production with \tsc{Sherpa}~\cite{Gleisberg:2008ta}.

The structure of this paper is as follows. 
In Section~\ref{sec:calculation}, we describe our different
calculations performed to compare theoretical
descriptions of the complex final state of two charged leptons, two
$b$-jets and missing energy. In Section~\ref{sec:results}, we 
compare these different theoretical descriptions for a number of
observables relevant to top quark mass measurements. 
We then quantify in Section \ref{sec:fit} how the differences in
the theoretical descriptions impact a template fit as utilised
in experimental determinations
of the top quark mass, before we conclude in Section~\ref{sec:conclusions}.
