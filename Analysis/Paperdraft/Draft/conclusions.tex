We have studied the impact of various theoretical descriptions for top
quark pair production on measurements of the top quark mass in the
di-lepton channel. In particular, we have compared the NLO QCD results
for $W^+W^- b\bar{b}$ production ($\nlofull$) to results based on the narrow-width
approximation, combining $t\bar{t}$ production at NLO with ($i$) LO top
quark decays ($\lodec$), ($ii$) NLO top quark decays
($\nlodec$) and ($iii$) a parton shower ($\nlops$). 
%
We have assessed the theoretical uncertainties associated with the
different theory descriptions via the variation of renormalisation,
factorisation and shower scales, and investigated the top quark mass
sensitivity of the observables $\mlb$, $\mtwo$, $\mll$ and $\etdr$.

Based on these results, we then studied the prospects of a top quark
mass extraction from the observables $\mlb$ and $\mtwo$, which we
found to be most sensitive to top quark mass variations.
Using pseudo-data based on our calculations, we employed the template
method to determine the offset in the top quark mass from calibrations
that differ in their underlying theory description. These analyses
show that the behaviour of the observables $\mlb$ and $\mtwo$ is
rather similar in what concerns the observed offsets in the top quark
mass.

More importantly, we found that the NLO corrections to the top quark decay
play a significant role, because they lead to non-uniform scale
uncertainty bands.
%
As the fits are based on normalised differential cross sections, 
shape differences induced by the scale variations will 
lead to larger theory uncertainties for the top quark mass extraction.
%
Even though the total scale uncertainties decrease at NLO as to be expected, 
the shape changes on the $\mlb$ distribution induced by scale variations are
particularly pronounced in the cases where the decay is described at NLO.
%
For both the $\nlofull$ as well as the $\nlodec$ description, the theoretical
uncertainties in determining \mt\ therefore increase by at least a
factor of two compared to the uncertainties emerging when LO decays are
involved.
%
Furthermore, the direct comparison of theories differing in their treatment
of the top quark decays can lead to offsets of more than $1\gev$ in the
measured \mt value.
%
This is observed in both cases, i.e.~when confronting $\nlofull$ pseudo-data
with the $\lofull$ calibration and $\nlodec$ pseudo-data with the $\lodec$
calibration.
%
These findings indicate that the non-resonant and non-factorising contributions
have a smaller effect on the top quark mass extraction than the NLO treatment
of the decay.

Turning to the parton shower ($\nlops$) results of our analysis
approach, we have compared them to the theory models $\nlofull$ and
$\nlodec$, leading to mass shifts of $-0.09\pm0.07\gev$ and
$0.96\pm0.07\gev$, respectively (in the \mlb\ case).
%
The good agreement between $\nlofull$ and $\nlops$ results can be
attributed to the fact that the two descriptions are rather similar
for an appropriate fit range, but it does not mean that the two descriptions
agree for the entire $\mlb$ range. Resummation effects for low $\mlb$ values
in the $\nlops$ case and off-shell effects affecting the tail in the
$\nlofull$ case are clearly visible in the $\mlb$ distribution.
The differences between $\nlops$ and $\nlodec$ mainly originate from
the regions of small and near-edge $\mlb$ values, where resummation
corrections play an important role.

To better understand these differences, we investigated the parton shower
behaviour in more detail.
%
We considered results where we limit the number of emissions in both the
production and the decay showers, and indeed observe that the predictions of
such restricted parton showers move closer to the fixed-order $\nlodec$ result.
%
These investigations also showed that the resummation corrections incorporated
by the unrestricted showers may lead to effects on the top quark mass
determination that can be as large as $1\gev$.
%
In addition, we have switched off the shower emissions in either production or
decay, and found that both the production and the decay showers impact our
analysis in a significant manner.
%
Different ways to assess the shower scale uncertainties within the $\nlops$
description were also studied but their effect turned out to be small. 
%
The choice of a different central scale also had only a minor impact on the mass determination.

We finally investigated how the choice of the fit range impacts our
results and found that the corresponding offsets do not change considerably
if the fit range is altered (in a way that still leads to acceptable closure).

Based on our results, we expect that the non-uniform scale variation bands
in the $\mlb$ distribution, induced by NLO corrections to the decay as present
in the $\nlofull$ calculation, would not level out largely if a parton
shower was matched to $\nlofull$.
%
It is therefore conceivable that a top quark mass extraction based on LO
(or shower approximated) decays may underestimate the theoretical
uncertainties, even if higher perturbative orders in the top quark pair
production process are taken into account.

In the future, it would be very interesting to see how the pseudo-data
used here compare to real data. In this context, the impact of hadronisation
and colour reconnection effects should be studied.
%
Owing to the rather strong impact of the resummation, it would also be useful
to perform a dedicated comparison of different parton shower
prescriptions such as different evolution variables and recoil strategies.
%
Furthermore, it would be worthwhile to investigate how the NLO
results for the full $W^+W^- b\bar{b}$ final state, ideally matched to a
parton shower, compare to NNLO results for top quark pair production in the
narrow-width approximation, combined with different descriptions of the top
quark decay.

